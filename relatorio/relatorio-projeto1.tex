\documentclass[10pt]{article}

\usepackage[portuguese]{babel}
\usepackage[utf8]{inputenc}
\usepackage[left=2cm, right=2cm, top=2cm, bottom=2cm]{geometry}
\usepackage{graphicx}
\usepackage{underscore}
\usepackage{amsmath}
\usepackage{caption}
\usepackage{textcomp}
\usepackage{float}
\usepackage{algorithm}
\usepackage[noend]{algpseudocode}
\usepackage{listings}
\lstdefinestyle{python}{
  basicstyle=\footnotesize,
  language=Python,
  numbers=none,
  stepnumber=1,
  numbersep=1pt,
  tabsize=4,
  showspaces=false,
  showstringspaces=false
}

\begin{document}
\title{Relatório - Projeto 1\\
 	   MC322 - Programação Orientada a Objetos\\
 	   \textbf{Assistência e mapeamento do COVID-19 por municípios}
}
 	   
\author{\textbf{Grupo 13:}\\
		Airton Cardoso Lana \ \ \ \ RA: 212234\\
		Caio Vinicius Castro dos Santos \ \ \ \ RA: 214188\\
		Ian Loron de Almeida \ \ \ \ RA: 198933\\
		Murilo Gonçalves \ \ \ \ RA: 203904\\
}
\date{}

\maketitle

\section{Resumo}
\hspace{2em} Neste projeto objetiva-se criar um sistema que ofereça estrutura para o mapeamento e prevenção da Covid-19 a nível municipal, além de orientar os cidadãos que tiverem algum sintoma relacionado à doença a buscar a orientação médica mais próxima. 

\hspace{2em} Nesse contexto, o sistema deve fornecer uma pré-avaliação, baseada nos sintomas descritos pelo cidadão cadastrado, indicando a frequência e gravidade de seu quadro. Dessa forma, a partir de sua localização na cidade, é possível indicar qual o hospital mais próximo dessa pessoa e a partir do diagnóstico realizado, o cidadão também tornar-se um paciente para o sistema.

\hspace{2em} Entretanto, somente a localização não é suficiente para indicar o local mais adequado para uma pessoa buscar ajuda, assim, também são levados em conta fatores do cidadão, como se possui convênio, e fatores do hospital, por exemplo, se há disponibilidade de leitos no momento.

\hspace{2em} Além disso, também há a implementação da secretaria de saúde, responsável por adicionar uma nova cidade no sistema ou alterar o número de hospitais do município, caso um novo seja criado ou haja a desativação de algum hospital existente.

\hspace{2em} Assim, o mapeamento dos cidadãos será importante para indicar concentração de infectados, pessoas que estão saudáveis e pessoas que se recuperaram, possibilitando uma distribuição de recursos e insumos muito melhor, além de evitar óbitos ou casos mais graves da doença através de diagnóstico precoce, já que o cidadão não perderá tempo indo em um hospital muito longe ou em que não possa ser atendido.

\section{Funções do Sistema}
O sistema deve ser capaz de:
\begin{itemize}
\item Cadastrar cidadão;
\item Cadastrar cidade;
\item Adicionar/Remover hospital da cidade;
\item Identificar os hospitais públicos e privados da cidade e identificar os planos de saúde que eles trabalham;
\item Contabilizar cidadãos com COVID-19;
\item Contabilizar número de casos por região da cidade;
\item Identificar suspeita de COVID-19 de um cidadão através dos sintomas que o Cidadão informar no sistema;
\item Identificar hospital mais próximo e com leitos disponíveis a um cidadão;
\item Retorna a gravidade dos sintomas que o Paciente tem;
\item Determinar se um cidadao é grupo de risco, de acordo com as informacoes fornecidas pelo mesmo;
\end{itemize}	



No entanto, os itens sublinhados serão implementados apenas na próxima fase.


\end{document}